\documentclass[11pt, oneside]{amsart}
\usepackage{geometry}                % See geometry.pdf to learn the layout options. There are lots.
\geometry{letterpaper}                   % ... or a4paper or a5paper or ... 
%\geometry{landscape}                % Activate for for rotated page geometry
\usepackage[parfill]{parskip}    % Activate to begin paragraphs with an empty line rather than an indent
\usepackage{graphicx}
\usepackage{amssymb}
\usepackage{epstopdf}
\usepackage{url}
\DeclareGraphicsRule{.tif}{png}{.png}{`convert #1 `dirname #1`/`basename #1 .tif`.png}

\title{TUG-TeXLive-2017}
\author{Richard Koch}
%\date{}                                           % Activate to display a given date or no date

\begin{document}
\maketitle
\section{TeXLive-2017}
This is the portion of the TUG documentation which deals with making the TeXDist package.
This is used in the DVD version of the Installer.

A clean package file will contain
\begin{itemize}
\item Files ``buildPackage.sh'' and ``signPackage.sh
\item Files  ``background.jpg'', ``License.rtf'', ``ReadMe.rtf'', and ``Welcome.rtf''
\item The main folder  has a scripts folder containing  files ``Description.rtf'', ``postinstall'', 
 ``texdist'', ``TeXDist-description.rtf'', ``TeXDistVersion'', ``TeXManPath'', and ``TeXPath''. 
\item Two project files for PackageMaker, TeXDistBuild, and ``TeXDistBuild copy''
\item ``TUG'' folder with this document
\end{itemize}

\section{Updating for a New Year}

To update the  package for a new year, edit ``buildPackage.sh'' to
reflect the new package names, which contain the year.  

Also edit the License.rtf, ReadMe.rtf, and Welcome.rtf files both at the root level of the folder and inside the DVD folder.
 
Edit the two postflight scripts appropriately, mainly by changing the year.
For instance, to change from 2016 to 2017, use Search and Replace to change the many,
many occurrences of 2016 to 2017 in these files.

\section{Creating the  Package}

Issue the command

\begin{verbatim}
     sudo sh buildPackage.sh
 \end{verbatim}
to build the root folder, containing /Library/TeX.


\section{Building TeXDist}

This step is optional, since TeXDist will be part of MacTeX for the DVD.. There are two copies of the TeXDIstBuild template. The copy
titled ``TeXDistBuild copy'' is there for backup in case the original template becomes corrupted, and the
material below just provides additional backup.

Edit TeXDistBuild, the Packagemaker project file, using the GUI interface, as follows.

\begin{enumerate}
\item Click the top item on the left. The right side will change to a view with three tabs. First select
the Configuration tab. Edit so
\begin{enumerate}
\item Title: TeXDist
\item User Sees: Easy Install Only
\item Install Destination: System volume
\item Certificate: Leave this blank. You might think that putting a certificate name here would correctly sign
the package, but experiments show that it does not. The package needs to be signed after it is constructed.
\item Description: Empty
\end{enumerate}
\item The Requirements and Actions tab entries for the top item can be left blank
\item Click the entry TeXDist in the Contents section on the left. The right side will change to a view with two tabs.
Select the Configuration tab. Edit so
\begin{enumerate}
\item Choice Name: TeXDist
\item Identifier: choice1
\item Initial State: Selected and Enabled
\item Destiination: Leave Blank
\item Tooltip: Leave Blank
\item Description: Leave Blank
\end{enumerate}
\item The Requirements tab entries for this contents item can be left blank
\item Open the TeXDiste  entry in the bottom left column. This item should show a ``Library'' directory,
obtained by dragging root/Library to the left column. The right side will change to a view with four tabs.
 It is not necessary to edit the
items under the Contents or Components tabs. The items under the
Configuration tab should be edited as follows:
\begin{enumerate}
\item Install: root/Library (The item on the right should have selected ``Relative To Project'')
\item Destination: /Library (The item on the right should have selected ``Relative To Project'')
\item Do not check ``Allow custom location''
\item Patch: No patch root selected
\item Package Identifier: org.tug.texdist2017
\item Package Version: 1.0
\item Restart Action: None
\item Check ``Require admin authentication''
\item PackageLocation: Self-Contained
\end{enumerate}
The items under the Scripts tab should be edited to read
\begin{enumerate}
\item Scripts directory: scripts (The item on the right should have selected ``Relative To Project'')
\item Preinstall: Leave Blank
\item Postinstall: scripts/postinstall (The item on the right should have selected ``Relative To Project'')
\end{enumerate}

\end{enumerate}



\section{Signing}

Sign the package using the single command
\begin{verbatim}
     sh signPackage.sh
\end{verbatim}








\end{document} 



